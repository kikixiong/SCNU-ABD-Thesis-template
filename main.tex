% Author:Jiaqi Xiong
% Email:yoshiki.xiong@outlook.com

\documentclass[UTF8,AutoFakeBold=1,AutoFakeSlant,zihao=-4]{SCNU}

% Title and information
\newcommand{\thesisTitle}{Your Thesis Title}
% \newcommand{\thesisTitleEN}{Automated High-Rise Warehouse; Stacking Crane; Design}

\newcommand{\yourDept}{Aberdeen Institute of Data Science and Artificial Intelligence}
\newcommand{\yourMajor}{人工智能}
\newcommand{\yourName}{momo}
\newcommand{\yourClass}{\yourMajor 1班}
\newcommand{\yourMentor}{momo}
\newcommand{\studentID}{20213801000}

\usepackage{lipsum} % for dummy text

\begin{document}

% Cover Page (automatically generated)
\coverpage

% --- English Abstract ---
\begin{abstractEN}
\noindent\hspace*{4\ccwd}The abstract text starts with 4-character indentation. This is the first paragraph demonstrating proper indentation. The content should be written in Times New Roman font (small 4 size) with consistent formatting throughout.

The second paragraph maintains the same indentation style. Notice that the indentation is automatically applied without manual spacing commands. The key words section follows the specified format.

\keywordsEN{Deep Learning; Computer Vision; Natural Language Processing}
\end{abstractEN}

% ---  Chinese Abstract ---
% If you are required to include a Chinese abstract, uncomment the following:
\begin{abstract}
\noindent\hspace*{4\ccwd}中文摘要内容, 用小四号宋体字体,每段开头留4个字符空格,英文摘要的内容应与中文摘要基本相对应

“关键词”空两格,后加冒号与关键词隔开,各关键词之间用逗号隔开。中文关键词应与英文关键词相对应。关键词一般在 3—8 个之间。

\keywords{深度学习,计算机视觉,自然语言处理}
\end{abstract}

% --- Table of Contents ---
\contentpage
% --- PREFACE ---
\section*{PREFACE}



\newpage

% --- MAIN BODY ---
\section{MAIN BODY}
\subsection{First Section Title}
\subsubsection{Subsection example}
\vspace{1em}
\noindent\hspace*{4\ccwd}This is the first paragraph of the first section. All paragraphs should begin with an indent corresponding to 4 characters. 



\subsection{Example}

\subsubsection{Figure Example}
\begin{figure}[ht]
    \centering
    \includegraphics[scale=0.3]{figures/abd.png}
    \caption{University of Aberdeen.}
    \label{fig:example}
\end{figure}

\subsubsection{Table Example}
\begin{table}[ht]
    \centering
    \caption{ Feature Description}
    \begin{tabular}{lll}
    \toprule
                     & Feature Description & Dimension \\ \midrule
    Zero Crossing  & Frequency of waveform crossing the x-axis  & 1 \\
    Spectral Centroid & Indicates the “center of mass” of the sound & 1 \\ \bottomrule
    \end{tabular}
    \label{tab:example}
\end{table}
\subsubsection{Citation Example}

The transformer architecture, first introduced by \cite{vaswani2017attention}, revolutionized the field of natural language processing. As shown in Equation \ref{eq:loss}, the self-attention mechanism...

\subsubsection{Equation Example}

\begin{equation}
    loss = -\sum_{i=1}^{n}{y_i\,\log(\hat{y_i})} + \lambda \left\lVert\omega\right\rVert_2^2
    \label{eq:loss}
\end{equation}

% --- Code Example ---
\begin{verbatim}
import numpy as np

def normalize(x):
    mean = np.mean(x)
    std = np.std(x, ddof=1)
    return (x - mean) / std
\end{verbatim}




% --- CONCLUSION ---

\section{Conclusion}
\subsection{Main result}
\noindent\hspace*{4\ccwd}Briefly review the work completed, the research method used, the experiments or simulations conducted, and the results obtained. Do not completely copy the abstract; the content should be similar but not identical.




% --- REFERENCES ---

\begin{references}
    \bibliography{references.bib}
\end{references}

% --- APPENDIX ---
\clearpage
\nonumsection{APPENDIX}
\appendixformat
\noindent\hspace*{4\ccwd}Include lengthy derivations, full code listings, or supplementary tables and figures as needed.

\end{document}
